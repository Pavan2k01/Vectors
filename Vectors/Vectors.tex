\documentclass[12pt]{article}
\usepackage[none]{hyphenat}
\usepackage{amsmath}
\usepackage{graphicx}
\usepackage{atbegshi}
\AtBeginDocument{\AtBeginShipoutNext{\AtBeginShipoutDiscard}}
\newcommand{\solution}{\noindent \textbf{Solution: }}
\providecommand{\brak}[1]{\ensuremath{\left(#1\right)}}
\newcommand{\myvec}[1]{\ensuremath{\begin{pmatrix}#1\end{pmatrix}}}
\let\vec\mathbf
\begin{document}
\graphicspath{{./Documents}{./figs}}
\begin{center}
	\title{\textbf{Vector}}
	\date{\vspace{-5ex}}
	\maketitle
\end{center}
\setcounter{page}{1}
\section*{11$ ^{th} $ Maths - Chapter 10}
The following problem is question 09 from exercise 10.4:
\begin{enumerate}
	\item Find the value of $\vec{p}$ so that the three lines $ 3x + y - 2 = 0, px + 2y - 3 = 0 $ and $ 2x - y - 3 = 0 $ may intersect at one point.
\end{enumerate}
\solution \\
Given equations can be written in the form of $ \vec{a}^{T}\vec{x} = c$ \\
Therefore, \\ 
\begin{align}
	\myvec{3&1}\vec{x}=2
\end{align}
\begin{align}
	\myvec{p&2}\vec{x}=3
\end{align}
\begin{align}
	\myvec{2&-1}\vec{x}=3
\end{align}\\
Now, Solving equations (1) and (3)\\augumented matrix is
\begin{align}
	\begin{pmatrix}
		3 & 1 & 2 \\ \\
		2 & -1 & 3
	\end{pmatrix}
\end{align}
$ R_1 \rightarrow R_1 + R_2 $
\begin{align}
	\begin{pmatrix}
		5 & 0 & 5 \\ \\
		-2 & 1 & -3
	\end{pmatrix}
\end{align}
$ R_1 \rightarrow \frac{R_1}{5} $
\begin{align}
	\begin{pmatrix}
		1 & 0 & 1 \\ \\
		-2 & 1 & -3
	\end{pmatrix}
\end{align}
$ R_2 \rightarrow 2R_1 + R_2 $
\begin{align}
	\begin{pmatrix}
		1 & 0 & 1 \\ \\
		0 & 1 & -1
	\end{pmatrix}
\end{align}
Therefore, $ \vec{x} = \myvec{1 \\ \\ -1} $ If the three lines may intersect, this point lies on equation(2), then\\
\begin{center}
	$ \myvec{p & 2}\myvec{1 \\ \\ -1} = 3 $
	\vspace{\baselineskip}\\
	$ p-2 = 3 $
	\vspace{\baselineskip}\\
	$ p = 5 $
	\vspace{\baselineskip}\\
	Therefore, the equation is $ \myvec{5 & 2}\vec{x} = 3 $
\end{center}
\begin{figure}
	\centering
	\includegraphics[width=\columnwidth]{figs/graph.jpg}
	\caption{Straight-lines}
	\label{fig:st.lines}
\end{figure}
\end{document}
